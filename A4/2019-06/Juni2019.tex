% compile with XeLaTex for font support!
% Font licenses:
% OpenSans published under Apache License Version 2.0, for more details see \ressources\fonts\open-sans\Apache License.txt or https://www.apache.org/licenses/LICENSE-2.0
% PoliticsHead by Fred Bordfeld published under Creative Commons Attribution-NonCommercial-NoDerivatives 4.0 International Public License, for more details see \ressources\fonts\PoliticsHead\CC_by_nc_nd Licence.txt or https://creativecommons.org/licenses/by-nc-nd/4.0/legalcode

\documentclass[flyer]{netz39} % DIN A4, tri-fold
%\documentclass[poster]{netz39} % DIN A3, one-sided

\begin{document}
	
\title{Veranstaltungsübersicht}{Juni/Juli 2019}

\begin{about}
	\section{Was ist der Netz39 e.V.?}
Netz39 e.V. ist ein gemeinnütziger Verein, in dem es ums Selbermachen und Reparieren und ganz allgemein um Spaß an Technik geht. Wir machen alles Mögliche, von 3D-Druck über Elektronik, Programmieren, unsere Fahrräder flicken (viel zu oft), Möbel bauen bis Nähen. 

Man darf einfach reinkommen und sich das anschauen! Wirklich! Es gibt keine regelmäßigen Öffnungszeiten, aber sehr oft ist ab nachmittags bis abends jemand da. Man kann auch auf der Website sehen, ob offen ist.

Die Räumlichkeiten und Geräte werden von den Mitgliedern des Vereins finanziert. Gäste können die Infrastruktur nach Absprache nutzen, eine Mitgliedschaft ist aber gerne gesehen, um die Räumlichkeiten und Ausstattung langfristig zu erhalten und auszubauen. Der monatliche Beitrag liegt für Vollzahler bei 30 Euro und für Studenten etc. bei 10 Euro. Wir freuen uns natürlich auch über Fördermitgliedschaften und Spenden!

Auch alle Veranstaltungen sind grundsätzlich offen für Gäste.
\end{about}

\begin{contact}
	\section{Kontakt}
	\textbf{E-Mail:} kontakt@netz39.de\\
	\textbf{Chat (XMPP):} lounge@conference.jabber.n39.eu\\
	\textbf{Twitter:} @netz39\\
	\textbf{WWW:} http://www.netz39.de\\
	\textbf{RealLife:} Netz39 e.V.\\
	Leibnizstraße 32\\
	39104 Magdeburg\\\\
	V.i.S.d.P.: Tatjana Ruhl
\end{contact}


\begin{entry}{Veranstaltungen}
siehe auch www.netz39.de/events
\end{entry}

\begin{entry}{Repaircafé}
Etwa einmal im Quartal veranstalten wir ein sogenanntes Repaircafé. Ihr könnt mit Euren kaputten Dingen, Sachen und Geräten zu uns kommen und wir versuchen sie gemeinsam zu reparieren. Das kannst du aber auch unabhängig vom RepairCafé bei uns nach Absprache tun.
\begin{events}{Nächste Termine}
	\event 02. Juni 2019 (13:00 bis 18:00 Uhr)
\end{events}
\end{entry}

\begin{entry}{Freifunk}
Das Projekt Freifunk Madeburg baut frei zugängliches WLAN auf, genauer gesagt schafft es die technischen Voraussetzungen dafür (Stichwort Firmware, Gateways etc.). Knoten aufstellen darf natürlich jeder. Das Team trifft sich alle zwei Wochen freitags zum Basteln und Organisieren.
\begin{events}{Nächste Termine}
	\event 14. Juni 2019 (19:30 Uhr)
	\event 28. Juni 2019 (19:30 Uhr)
\end{events}
\end{entry}


\begin{entry}{Workshops}
Es gibt immer wieder Workshops zu den unterschiedlichsten Dingen. Unsere letzten größeren Workshops waren zum Beispiel Platinendesign mit KiCad, Makrofotografie, Hoodie-Nähen und Musikinstrumente bauen. Das Konzept ist aber immer ähnlich: Selbst austoben mit eigenen Projekten unter Anleitung, weniger vorgefertigte Dinge zum Abarbeiten. Aber immer einsteigerfreundlich!
\begin{events}{Nächster Workshop}
	\event git für Fortgeschrittene (30. Juni 2019)
\end{events}
\end{entry}

\begin{entry}{Linux-Stammtisch}
Alle sechs Wochen sind alle interessierten Linux-Nutzer*innen und solche, die es werden wollen, in den Hackerspace eingeladen, um sich persönlich vor Ort über Themen rund um Linux auszutauschen. Es gibt üblicherweise ein übergreifendes Thema pro Treffen, das etwa eine Woche vorher festgelegt wird. 
\begin{events}{Nächster Termin}
	\event 29. Mai 2019 (19:30 Uhr)
	\event 10. Juli 2019 (19:30 Uhr)
\end{events}
\end{entry}

\begin{entry}{TechTalks}
Immer am vierten Montag im Monat gibt es einen Talk zu einem nerdigen Thema mit anschließender Diskussion. Start jeweils 19:30 Uhr.
\begin{events}{Nächste Termine}
	\event 27. Mai 2019 (Mit Kryptographie gegen Steuerbetrug)
	\event 24. Juni 2019 (tba)
	\event 22. Juli 2019 (tba)
\end{events}
\end{entry}

% \begin{entry}{IT-Bastel-Runde - Gemeinsam programmieren (lernen)}
% Ihr habt Ideen für ein Softwareprojekt und wisst nicht wie ihr anfangen sollt oder ihr wollt euch mal an einem Softwareprojekt ausprobieren und sucht nach anderen Leuten, mit denen ihr das gemeinsam bewältigen wollt? Dann kommt doch zu unserer „IT-Bastel-Runde“, wo man sich gemeinsam hilft und austauscht. Erfahrene Programmierer unterstützen euch. Die Gruppe startet demnächst neu.
% \begin{events}{Erstes konspiratives Treffen}
% 	\event tbd, bei Interesse bitte Mail an kontakt@netz39.de
% \end{events}
% \end{entry}

\begin{entry}{Plenum}
Unser dreiwöchentliches Orgatreffen zur Diskussion von Fragen aller Art. Auch zum „Einfach-Mal-Reingucken“ für Gäste geeignet. 
\begin{events}{Nächste Termine} 
	\event 12. Juni 2019 (19:30 Uhr)
	\event 03. Juli 2019 (19:30 Uhr)
\end{events}
\end{entry}

\begin{entry}{Freies Hacken}
An den eigenen Projekten basteln kann man bei uns rund um die Uhr. Damit man ein bisschen planen kann, wenn man dabei Gesellschaft haben möchte, gibts den Mittwochs-Termin „Freies Hacken“. Immer, wenn kein Plenum oder Linux-Stammtisch ist. Immer ab etwa 19 Uhr. Auch eine gute Gelegenheit, um einfach mal reinzuschauen.
\begin{events}{Nächste Termine} 
	\event 05. Juni 2019 (19:00 Uhr)
	\event 19. Juni 2019 (19:00 Uhr)
\end{events}
\end{entry}

\begin{entry}{Sprechstunde bei Dr. Space}
Ihr habt ein Problem mit eurem Rechner/Smartphone, wollt als Gast mal unser Werkzeug benutzen oder braucht ansonsten Rat von Dr. Space? Dann kommt nach dem Plenum rum und wir schauen, was wir tun können. 
\begin{events}{Nächste Termine} 
	\event 12. Juni 2019 (21:00 Uhr)
	\event 03. Juli 2019 (21:00 Uhr)
\end{events}
\end{entry}

\begin{entry}{Preview: CCCamp}
Natürlich sind wir auch dieses Jahr wieder beim großen Hackercamp des CCC in Mildenberg dabei. 
\begin{events}{} 
	\event 21. bis 25. August 2019
\end{events}
\end{entry}

%\begin{entry}{Linux Presentation Day}
%Ein bis zwei Mal im Jahr stellen wir das freie Betriebssystem und freie Anwendungen vor. Neugierige Nutzer können zuhören, ausprobieren, fragen, installieren…
%\emph{Der LPD 2018.2 wird am 10. November 2018 stattfinden.}
%\end{entry}

%\begin{entry}{35C3}
%Üblicherweise fährt eine größere Gruppe Leute von uns zum Kongress des CCC, mit Assembly und so. Die Vorbereitungen laufen so langsam an. Der Kongress findet dieses Jahr wieder in Leipzig statt, wie immer vom 27. bis 31. Dezember. 
%\end{entry}

%\begin{entry}{TutOrials}
%Bei Bedarf machen wir Kurzworkshops, in denen wir erklären, wie die verschiedenen Maschinen funktionieren, insbesondere der 3D-Drucker und die CNC-Fräse. Aber natürlich auch das jeweils neu angeschaffte Spielzeug :)
%\end{entry}

\end{document}